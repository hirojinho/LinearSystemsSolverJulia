\documentclass[aspectratio=169]{beamer}

\usepackage[utf8]{inputenc}
\usepackage[brazil]{babel}
\usepackage{amsmath}
\usepackage{amssymb}
\usepackage{graphicx}
\usepackage{booktabs}
\usepackage{algorithm}
\usepackage{algorithmic}

% Tema
\usetheme{Madrid}
\usecolortheme{seahorse}

% Informações do título
\title{Decomposição em Modos Dinâmicos (DMD)}
\subtitle{Teoria Computacional e Conexões com Métodos de Álgebra Linear}
\author{Eduardo Hiroji Brandão Haraguchi}
\institute{Instituto Tecnológico de Aeronáutica\\MAT-55}
\date{\today}

\begin{document}

% Slide de título
\begin{frame}
    \titlepage
\end{frame}

% Slide de overview
\begin{frame}{Roteiro da Apresentação}
    \tableofcontents
\end{frame}

\section{Introdução e Motivação}

\begin{frame}{Introdução}
    \begin{itemize}
        \item \textbf{DMD (Dynamic Mode Decomposition)}: Extensão computacional de métodos de álgebra linear
        \item Desenvolvida por Schmid (2010)
        \item Aplicação sistemática de conceitos estudados em MAT-55:
        \begin{itemize}
            \item Decomposição SVD
            \item Métodos de quadrados mínimos
            \item Técnicas de ortogonalização
        \end{itemize}
        \item \textbf{Objetivo}: Extrair informações dinâmicas de dados temporais
    \end{itemize}
\end{frame}

\begin{frame}{Motivação}
    \begin{block}{Por que DMD?}
        \begin{itemize}
            \item Sistemas dinâmicos complexos geram dados temporais
            \item Necessidade de identificar padrões dinâmicos
            \item Métodos clássicos de álgebra linear são a base
            \item Robustez numérica herdada dos métodos fundamentais
        \end{itemize}
    \end{block}
\end{frame}

\section{Teoria Computacional da DMD}

\begin{frame}{Formulação Matricial do Problema}
    \begin{columns}
        \column{0.6\textwidth}
        Dados dois conjuntos de observações temporais:
        \begin{align}
            X &= [x_1, x_2, \ldots, x_{m-1}] \in \mathbb{R}^{n \times m}\\
            Y &= [x_2, x_3, \ldots, x_m] \in \mathbb{R}^{n \times m}
        \end{align}
        
        \textbf{Objetivo}: Encontrar $A \in \mathbb{R}^{n \times n}$ tal que:
        \begin{equation}
            Y \approx AX
        \end{equation}
        
        \column{0.4\textwidth}
        \begin{block}{Exemplo}
            \begin{itemize}
                \item $x_1, x_2, \ldots, x_m$: snapshots temporais
                \item $X$: dados no tempo $t$
                \item $Y$: dados no tempo $t+1$
                \item Relação linear: $Y \approx AX$
            \end{itemize}
        \end{block}
    \end{columns}
    
    \vspace{0.5cm}
    \begin{block}{Conexão com MAT-55}
        Problema de quadrados mínimos: $Ax \approx b$
    \end{block}
\end{frame}

\begin{frame}{Decomposição SVD como Base}
    \begin{block}{Aplicação da SVD}
        \begin{equation}
            X = U\Sigma V^T
        \end{equation}
        onde $U \in \mathbb{R}^{n \times n}$ e $V \in \mathbb{R}^{m \times m}$ são ortogonais
    \end{block}
    
    \begin{block}{Truncagem SVD}
        \begin{equation}
            X \approx U_r\Sigma_r V_r^T
        \end{equation}
        Mantendo apenas os $r$ primeiros valores singulares
    \end{block}
    
    \begin{alertblock}{Importância}
        \begin{itemize}
            \item Redução de dimensionalidade
            \item Controle de condicionamento numérico
            \item Base para estabilidade do algoritmo
        \end{itemize}
    \end{alertblock}
\end{frame}

\begin{frame}{Construção do Operador DMD Reduzido}
    \begin{block}{Operador de Dimensão Completa}
        \begin{equation}
            A = YX^{\dagger} = YV_r\Sigma_r^{-1}U_r^T
        \end{equation}
        onde $X^{\dagger}$ é a pseudoinversa de Moore-Penrose
    \end{block}
    
    \begin{block}{Operador Reduzido}
        Para $n$ grande, trabalhamos com:
        \begin{equation}
            \tilde{A} = U_r^T A U_r = U_r^T Y V_r \Sigma_r^{-1}
        \end{equation}
        onde $\tilde{A} \in \mathbb{R}^{r \times r}$ com $r \ll n$
    \end{block}
    
    \begin{itemize}
        \item \textbf{Dupla redução dimensional}: espacial e temporal
        \item Preserva propriedades espectrais essenciais
        \item Eficiência computacional
    \end{itemize}
\end{frame}

\section{Algoritmos DMD}

\begin{frame}{Algoritmo DMD Clássico}
    \begin{algorithm}[H]
        \caption{DMD Clássico}
        \begin{algorithmic}[1]
            \REQUIRE $X, Y \in \mathbb{R}^{n \times m}$, rank $r$
            \STATE Aplicar SVD: $X = U\Sigma V^T$
            \STATE Truncar: $U_r, \Sigma_r, V_r$
            \STATE Construir: $\tilde{A} = U_r^T Y V_r \Sigma_r^{-1}$
            \STATE Resolver: $\tilde{A}W = W\Lambda$
            \STATE Modos: $\Phi = U_r W$
            \STATE Amplitudes: $b = \Phi^{\dagger} x_1$
            \RETURN $\Phi, \Lambda, b$
        \end{algorithmic}
    \end{algorithm}
    
    \begin{itemize}
        \item Modos residem no subespaço gerado por $U_r$
        \item Eficiente computacionalmente
        \item Base para variantes mais sofisticadas
    \end{itemize}
\end{frame}

\begin{frame}{Algoritmo DMD Exato}
    \begin{block}{Diferença Principal}
        \textbf{DMD Clássico}: $\Phi = U_r W$ (projeção no subespaço de $X$)
        
        \textbf{DMD Exato}: $\Phi_{\text{exact}} = \frac{1}{\lambda} Y V_r \Sigma_r^{-1} W$ (imagem de $Y$)
    \end{block}
    
    \begin{block}{Relação Matemática}
        \begin{equation}
            \Phi = \mathbb{P}_X \Phi_{\text{exact}}
        \end{equation}
        onde $\mathbb{P}_X = UU^T$ é a projeção ortogonal
    \end{block}
    
    \begin{itemize}
        \item DMD Exato utiliza toda a informação disponível
        \item Maior precisão na representação dos modos
        \item Autovetores verdadeiros do operador $A = YX^+$
    \end{itemize}
\end{frame}

\section{DMD vs PCA}

\begin{frame}{Comparação: DMD vs PCA}
    \begin{table}[h]
        \centering
        \small
        \begin{tabular}{|l|l|l|}
            \hline
            \textbf{Aspecto} & \textbf{PCA} & \textbf{DMD} \\
            \hline
            Ortogonalidade & Modos ortogonais & Não necessariamente \\
            \hline
            Ordenação & Por variância & Por relevância dinâmica \\
            \hline
            Info. temporal & Não captura & Frequências e crescimento \\
            \hline
            Interpretação & Estruturas energéticas & Dinâmicas coerentes \\
            \hline
            Reconstrução & $x_k = \Phi\alpha_k$ & $x_k = \sum_j \phi_j \lambda_j^k b_j$ \\
            \hline
        \end{tabular}
    \end{table}
    
    \begin{block}{Base Comum}
        Ambos utilizam SVD, mas com objetivos distintos:
        \begin{itemize}
            \item \textbf{PCA}: Decomposição espacial ótima
            \item \textbf{DMD}: Decomposição temporal-espacial
        \end{itemize}
    \end{block}
\end{frame}

\section{Fundamentação Teórica}

\begin{frame}{Conexão com o Operador de Koopman}
    \begin{block}{Operador de Koopman}
        Para sistema $x_{k+1} = F(x_k)$, o operador $\mathcal{K}$ atua sobre funções:
        \begin{equation}
            \mathcal{K}g(x) = g(F(x))
        \end{equation}
    \end{block}
    
    \begin{block}{Aproximação DMD}
        \begin{equation}
            \tilde{A} \approx \mathcal{K}|_{\text{span}\{x_1, x_2, \ldots, x_m\}}
        \end{equation}
    \end{block}
    
    \begin{itemize}
        \item Modos DMD $\approx$ eigenfunctions de Koopman
        \item Autovalores DMD $\approx$ eigenvalues de Koopman
        \item \textbf{Implicação}: DMD funciona mesmo para sistemas não-lineares
        \item Fundamento teórico rigoroso
    \end{itemize}
\end{frame}

\section{Conclusões}

\begin{frame}{Resumo do Trabalho}
    \begin{block}{Aspectos Teóricos}
        \begin{itemize}
            \item Formulação rigorosa do problema DMD
            \item Conexão sistemática com métodos de MAT-55
            \item Análise detalhada dos aspectos algorítmicos
        \end{itemize}
    \end{block}
    
    \begin{block}{Aspectos Computacionais}
        \begin{itemize}
            \item Algoritmos implementáveis com técnicas do curso
            \item Considerações de estabilidade numérica
            \item Variantes algorítmicas (Clássico vs Exato)
        \end{itemize}
    \end{block}
    
    \begin{block}{Fundamentação Matemática}
        \begin{itemize}
            \item Conexão com operador de Koopman
            \item Justificativa teórica para aplicação a sistemas não-lineares
        \end{itemize}
    \end{block}
\end{frame}

\begin{frame}{Reflexões Finais}
    \begin{alertblock}{Elegância dos Métodos de Álgebra Linear}
        DMD exemplifica como conceitos clássicos se estendem naturalmente para problemas contemporâneos de análise de dados dinâmicos
    \end{alertblock}
    
    \begin{block}{Importância da Base Sólida}
        \begin{itemize}
            \item Compreensão de SVD, quadrados mínimos e ortogonalização é essencial
            \item Implementação eficiente requer domínio dos fundamentos
            \item Estabilidade numérica herdada dos métodos base
        \end{itemize}
    \end{block}
    
    \begin{center}
        \Large \textbf{Obrigado!}
        
        \vspace{0.5cm}
        Perguntas?
    \end{center}
\end{frame}

\end{document} 